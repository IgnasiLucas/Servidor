\documentclass[a4paper,12pt]{article}
\usepackage[utf8]{inputenc}
\usepackage[T1]{fontenc}
\usepackage[spanish]{babel}
\usepackage{booktabs}
\usepackage[colorlinks=true]{hyperref}
\title{Encuesta sobre servicios generales informáticos para la docencia}
\hyphenation{Python}

\begin{document}
\maketitle
En la Facultad de Ciencias Biológicas, cada vez es más habitual que en alguna
asignatura se utilice un lenguaje de programación, como R o Python,
para realizar ejercicios o prácticas con ordenador. Estas iniciativas individuales
están con frecuencia relacionadas con una concepción abierta de la educación y de
la ciencia \cite{Schlagwein2017}, y contribuyen a proporcionar habilidades
computacionales básicas a estudiantes de cualquier grado \cite{Vee2017}. Sin embargo,
la adopción de lenguajes de programación en nuestras aulas se encuentra con
dificultades y necesita un apoyo institucional por parte de la Universidad.

El objetivo de esta encuesta es compartir esta percepción con el profesorado de todos
los centros de la Universidad de Valencia (UV en adelante) y recoger el apoyo
necesario para solicitar la puesta en marcha de servicios informáticos generales
que faciliten la adopción de lenguajes de programación en la enseñanza de cualquier
materia y contribuyan a la educación abierta.

Los datos recogidos son completamente anónimos y no se compartirán con terceros.
Sólo serán publicados los estadísticos agregados, en un informe que podréis consultar
en la web... Si alguna persona está interesada en ayudar a coordinar una solicitud
inter-facultativa de los servicios informáticos generales propuestos, puede ponerse
en contacto con...

\begin{enumerate}
\item Centro al que pertenece:
\item Género:
\item Edat:
   \begin{enumerate}
   \item Menos de 31.
   \item 31-40.
   \item 41-50.
   \item 51-60.
   \item Más de 60.
   \end{enumerate}
\end{enumerate}

Marca el grado de conformidad con cada una de las afirmaciones siguientes, entre cero
(en completo desacuerdo) y cinco (completamente de acuerdo).

{\small
\begin{tabular}{|r|p{9cm}|c|c|c|c|c|c|}
\toprule
&&0&1&2&3&4&5\\
\midrule
4&Tengo buenos conocimientos de al menos un lenguaje de programación.&&&&&&\\
\midrule
5&La enseñanza de las materias en las que participo podría mejorar con la adopción de un lenguaje de programación.&&&&&&\\
\midrule
6&Participo o podría participar activamente en la docencia de prácticas con ordenador.&&&&&&\\
\midrule
7&Estaría dispuesto o dispuesta a recibir formación en la aplicación de la programación informática a la docencia.&&&&&&\\
\midrule
8&Las aulas de informática con las que contamos disponen de Linux y de todos los programas que necesitamos.&&&&&&\\
\midrule
9&La mayor parte de programas informáticos que enseñamos a utilizar son software libre.&&&&&&\\
\midrule
10&Los y las estudiantes pueden instalar fácilmente en sus ordenadores los programas informáticos que utilizamos en el aula.&&&&&&\\
\midrule
11&Conozco y utilizo con cierta frecuencia documentos dinámicos de Rmarkdown en el entorno de RStudio \cite{RStudio}.&&&&&&\\
\midrule
12&Conozco y utilizo con cierta frecuencia los cuadernos Jupyter (Jupyter Notebooks) \cite{Barba2019}.&&&&&&\\
\midrule
13&La UV debería ofrecer un servidor de RStudio para que estudiantes y profesorado pueda conectarse remotamente.&&&&&&\\
\midrule
14&La UV debería ofrecer un servidor JupyterHub de cuadernos Jupyter \cite{JupyterHub}.&&&&&&\\
\midrule
15&La UV debería ofrecer acceso por \textsf{ssh} a un servidor de cálculo para estudiantes.&&&&&&\\
\midrule
16&La UV debería alojar su propio servidor de repositorios de \textsf{git} (e.g., Gitlab) para estudiantes y profesorado \cite{Paderborn2023,PereiraBraga2023}.&&&&&&\\
\midrule
17&La UV debería mantener un nodo propio de una red social federada, como Mastodon, y ofrecer cuentas a toda la comunidad, igual que con el correo electrónico \cite{Brembs2023,mastodon}.&&&&&&\\
\bottomrule
\end{tabular}
}

Si quieres sugerir algún otro servicio o hacer algún comentario, puedes hacerlo aquí:
\vspace*{3cm}

\bibliography{../references}
\bibliographystyle{ieeetr}
\end{document}
