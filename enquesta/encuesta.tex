\documentclass[a4paper,12pt]{article}
\usepackage[utf8]{inputenc}
\usepackage[T1]{fontenc}
\usepackage[spanish]{babel}
\usepackage{booktabs}
\usepackage[colorlinks=true]{hyperref}
\title{Encuesta sobre servicios generales informáticos para la docencia}
\hyphenation{Python}

\begin{document}
\maketitle
Un aspecto importante de la transformación digital de la sociedad actual es el uso creciente
de los lenguajes de programación, tanto en ámbitos profesionales como educativos, a todos
los niveles y en cualquier área. En la docencia universitaria, lenguajes como Python o R
están pasando de ser objetivos del aprendizaje en cursos especializados a ser un recurso
didáctico o vehículo de enseñanza de cualquier materia, en cualquier grado. Parte de la
popularidad de estos lenguajes de programación es debida al hecho de ser expresados en
texto plano, un estándar universal fácilmente interpretable en cualquier sistema operativo.
Esta transparencia de los materiales didácticos elaborados en Python, R o en casi cualquier
otro lenguaje de programación facilita enormemente que sean compartidos, reutilizados y
actualizados de manera colectiva y con licencias abiertas, tanto por una comunidad docente
como por los y las estudiantes. Otras tecnologías basadas en el texto plano (\LaTeX, git,
markdown, Jupyter notebooks, etc.) contribuyen a dar un carácter abierto a la educación
superior y a desarrollar las habilidades computacionales básicas de los y las estudiantes.

Con la única finalidad de planificar una posible implementación de nuevos servicios
informáticos que faciliten la adopción de estas tecnologías, el Servicio de Informática
necesita conocer el grado de interés que suscitan. Por este motivo, os pedimos que
contestéis las siguientes preguntas, que no os ocuparán más de 5 minutos. Las respuestas
individuales serán completamente anónimas y no serán compartidas con terceros. Los estadísticos
de resumen serán publicados en la página web del Servicio de Informática.

\begin{enumerate}
\item Centro al que pertenece:
\item Categoría profesional:
\item Género:
\item Edat:
   \begin{enumerate}
   \item Menos de 31.
   \item 31-40.
   \item 41-50.
   \item 51-60.
   \item Más de 60.
   \end{enumerate}

\item Participas en la docencia de prácticas con ordenador?
   \begin{itemize}
   \item Sí.
   \item No.
   \item Ahora no, pero lo he hecho o podría hacerlo.
   \end{itemize}

   \begin{enumerate}
   \item Los programas que se usan en esas prácticas ¿son software libre?
      \begin{itemize}
      \item Sí.
      \item No.
      \end{itemize}

   \item ¿Los y las estudiantes tienen facilidad para instalar dichos programas en sus ordenadores?
      \begin{itemize}
      \item Sí.
      \item No.
      \item Depende del sistema operativo.
      \end{itemize}

   \item ¿Podrías decir cuáles son los principales programas utilizados en las prácticas con ordenador?
   \vspace*{1cm}

   \item ¿Las aulas de informática disponen de ese software instalado?
      \begin{itemize}
      \item Sí, en MS Windows.
      \item Sí, en Linux.
      \item No.
      \end{itemize}

   \item Comenta brevemente qué incidencias has tenido con estos programas, si ha habido alguna.
      \vspace*{1cm}
   \end{enumerate}

\item ¿Tienes conocimiento de algún lenguaje de programación?
   \begin{itemize}
   \item Sí.
   \item No.
   \end{itemize}

   \begin{enumerate}
   \item ¿Cuáles?
      \begin{itemize}
      \item Python.
      \item R.
      \item Julia.
      \item Go.
      \item C/C++.
      \item Perl.
      \item Pascal.
      \item Bash.
      \item Awk.
      \item Otros:
      \end{itemize}
   \end{enumerate}

\item ¿Conoces y utilizas el entorno RStudio \cite{RStudio}?
   \begin{itemize}
   \item Sí, lo conozco, pero no lo utilizo.
   \item Sí, lo utilizo, pero no en la docencia.
   \item Sí, lo utilizo, incluso para la docencia.
   \item No, ni lo conozco ni lo utilizo.
   \end{itemize}

   \begin{enumerate}
   \item Si lo haces, ¿conoces y utilizas los documentos dinámicos de tipo Rmarkdown?
      \begin{itemize}
      \item Sí, los conozco, pero no los utilizo.
      \item Sí, los utilizo, pero no en la docencia.
      \item Sí, los utilizo, incluso para la docencia.
      \item No, ni los conozco ni los utilizo
      \end{itemize}

   \item ¿Crees que la UV debería ofrecer un servidor de RStudio para su uso remoto en la docencia,
         por parte de estudiantes y profesorado?
      \begin{itemize}
      \item Sí.
      \item No.
      \item Indiferente.
      \end{itemize}
   \end{enumerate}

\item ¿Conoces y utilizas el entorno Jupyter (Jupyter notebooks o Jupyter lab) \cite{Barba2019}?
   \begin{itemize}
   \item Sí, lo conozco, pero no lo utilizo.
   \item Sí, lo utilizo, aunque no para la docencia.
   \item Sí, lo utilizo, incluso para la docencia.
   \item No, ni lo conozco ni lo utilizo.
   \end{itemize}

   \begin{enumerate}
   \item ¿Crees que la UV debería ofrecer un servidor Jupyter Hub para uso remoto en la docencia,
         por parte estudiantes y profesorado? \cite{JupyterHub}.
      \begin{itemize}
      \item Sí.
      \item No.
      \item Indiferente.
      \end{itemize}
   \end{enumerate}

\item ¿Qué ventajas crees que tendría sustituir una aplicación de interfaz gráfica por un \emph{script}
      (documento de texto plano ejecutable) de licencia abierta en la docencia de prácticas con ordenador?
   \begin{itemize}
   \item Reproducibilidad.
   \item Portabilidad a cualquier sistema operativo.
   \item Aprendizaje pasivo de un lenguaje de programación.
   \item Eliminación de \emph{cajas negras} en la enseñanza.
   \item Libertad de mejorar o adaptar el código a cada práctica concreta.
   \item Otras:
   % Podria ser més barat, si l'aplicació d'interfícia gràfica és comercial. Podria ser valorat el fet de què el mateix guió de práctiques és el programa amb què es fa la pràctica...
   \item Ninguna.
   \end{itemize}

\item ¿Qué desventajas o dificultades conllevaría sustituir una aplicación de interfaz gráfica por un
   \emph{script} de licencia abierta en la docencia de prácticas con ordenador?
   \begin{itemize}
   \item La necesidad de aprender el lenguaje de programación por parte del profesorado.
   \item El esfuerzo de tener que preparar la práctica de nuevo.
   \item El uso de una interfaz menos atractiva para los y las estudiantes.
   \item Otras:
   \item Ninguna.
   \end{itemize}

\item ¿Te gustaría recibir información sobre cómo aplicar los lenguajes de programación a la
      docencia?
   \begin{itemize}
   \item Sí.
   \item No.
   \end{itemize}

   \begin{enumerate}
   \item Si hay algún lenguaje de programación que te gustaría aprender, ¿cuál es?
   \vspace*{1cm}
   \end{enumerate}

\item ¿Conoces algún sistema de control de versiones (e.g., \textsf{git}, Subversion, Mercurial) y
      para qué sirven?
   \begin{itemize}
   \item Sí, pero no los uso.
   \item Sí, y uso alguno, aunque no en relación con la docencia.
   \item Sí, y uso alguno incluso con documentos destinados a la docencia.
   \item No.
   \end{itemize}

   \begin{enumerate}
   \item Siendo los materiales docentes documentos generalmente compartidos y actualizados con cierta
         frecuencia, ¿por qué razones crees que no suelen contar con un sistema de control de versiones?
      \begin{itemize}
      \item En realidad, solo se comparten ocasionalmente y no necesitamos harmonizar las versiones.
      \item Por desconocimiento de los sistemas de control de versiones por parte del profesorado.
      \item Porque la UV no ofrece ningún repositorio remoto donde mantener y compartir los materiales
            docentes.
      \item Porque los materiales docentes incluyen imágenes o fragmentos protegidos con licencias
            privativas que no pueden compartirse abiertamente.
      \item Porque generalmente usamos formatos privativos (MS Word, .ppt, etc.), que no pueden
            editarse en cualquier sistema operativo.
      \item Otras:
      \vspace*{1cm}
      \end{itemize}

   \item ¿Crees que mantener colectivamente los materiales docentes bajo un sistema de control de
         versiones sería deseable?
      \begin{itemize}
      \item Sí.
      \item No.
      \item Indiferente.
      \end{itemize}

   \item ¿Estarías dispuesto o dispuesta a recibir formación sobre el uso de un sistema de control de
         versiones como \textsf{git}?
      \begin{itemize}
      \item Sí.
      \item No.
      \end{itemize}

   \item ¿Crees que la UV debería tener su propio servidor de repositorios de \textsf{git}, como un
         Gitlab? \cite{Paderborn2023,PereiraBraga2023}
      \begin{itemize}
      \item Sí.
      \item No.
      \item Indiferente.
      \end{itemize}
   \end{enumerate}

\item En algunas asignaturas se usan o podrían usarse bases de datos grandes y/o una intensidad de
      computación elevada, que son difíciles de implementar en los ordenadores de un aula de
      informática. En tu docencia habitual, ¿encuentras algún motivo como estos para que estudiantes
      y profesorado tuvieran que conectarse a un servidor de cálculo remoto?
   \begin{itemize}
   \item Sí.
   \item No.
   \end{itemize}

   \begin{enumerate}
   \item Si es así, ¿puedes concretar qué recurso sería más accesible a través de un servidor de 
         cálculo remoto?
   \vspace*{1cm}
   \end{enumerate}
\end{enumerate}

\bibliography{../references}
\bibliographystyle{ieeetr}
\end{document}
