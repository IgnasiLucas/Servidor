\documentclass[a4paper,12pt]{article}
\usepackage[utf8]{inputenc}
\usepackage[T1]{fontenc}
\usepackage[catalan]{babel}
\usepackage[colorlinks=true]{hyperref}

\title{Resum de la reunió del 06/06/2023}

\begin{document}
\maketitle

Separem el temes a tractar en dues àrees, a pesar que algunes limitacions detectades
a les aules d'informàtica serien també corregides per l'existència de serveis generals
adequats.

\section{Aules d'informàtica}
\subsection{Ordinadors virtuals en Linux}
Celebrem la notícia de què a partir del pròxim curs als ordinadors de les aules hi haurà la possibilitat d'iniciar
un entorn virtual de Linux, on estiguen instal·lats els programes de Linux que hagem
sol·licitat. Això soluciona el problema de què no podíem sol·licitar la instal·lació
de programari en Linux.

En aquest entorn virtual les actualitzacions no seran més àgils del que ja ho són en
l'entorn de Windows. I en tractar-se de màquines virtuals, totes les modificacions
(descàrregues d'arxius, instal·lació de paquets en espai d'usuari...) desapareixeran
tan prompte com la màquina s'apague.

Quant al problema de què la màquina virtual no aprofitava la pantalla sencera,
ja està solucionat: hi ha un botó que permet ocupar la pantalla completa.

\subsection{Ordinadors físics, amb Ubuntu}
Existeix la possibilitat de treballar directament en el sistema operatiu
de l'ordinador de l'aula, que és Ubuntu. Actualment, en les màquines físiques no
hi ha instal·lats més que els programes essencials, perquè es consideren només una
porta d'entrada als ordinadors virtuals, on es manté el programari de docència. Així,
s'evita la saturació dels discs durs, etc. Però es podria instal·lar alguna cosa més,
per fer els ordinadors físics més útils. L'avantatge principal de poder treballar en
el sistema de la màquina física seria poder guardar la configuració de la persona usuària:
els documents creats, els paquets instal·lats en l'espai d'usuari, etc. Però aquest
avantatge només podríem gaudir-lo si cada estudiant treballa sempre en la mateixa màquina.

Aquesta possibilitat crec que va quedar en l'aire, si ho vaig entendre bé, en espera de
què pensem si realment val la pena. Jo pense que sí. Si els ordinadors
físics tingueren les eines bàsiques que qualsevol assignatura pot necessitar (R, RStudio,
Python, \LaTeX, JupyterLab, git i potser conda) entraríem ben poc a l'ordinador virtual,
almenys en algunes assignatures.

\section{Serveis informàtics generals}
Entre les propostes inicials hi havia alguns serveis informàtics que serien útils més
enllà de la Facultat. Entenem que per tal que el Servei d'Informàtica dedique recursos
a oferir aquests serveis, ha d'haver un interés més general. Ens comprometem a fer una
consulta i intentar arreplegar els suports necessaris perquè s'oferisquen aquests serveis.
Es va comentar també que si els centres oferiren part dels recursos necessaris per posar
en marxa una iniciativa així seria molt més probable que la Universitat assumira la seua
part.

Els serveis dels que estem parlant serien, en principi els següents:

\begin{description}
   \item[JupyterHub]\footnote{\href{https://jupyter.org/hub}{https://jupyter.org/hub}} Permetria
que ens connectàrem a través del navegador
a un servidor de càlcul, amb una interfície idèntica a la de JupyterLab, i que els
processos s'executaren allà, remotament. De fet, JupyterHub ofereix també altres interfícies:
com RStudio-server o consoles de Python, R o Bash, i per tant solucionaria més d'una
necessitat alhora. Entre les utilitats de JupyterHub per a docència, identifiquem les
següents:
   \begin{enumerate}
   \item L'ús de Jupyter Notebooks com a guions de pràctiques, que combinen explicació,
         dades i anàlisis de forma interactiva\footnote{\href{https://jupyter4edu.github.io/jupyter-edu-book/}{https://jupyter4edu.github.io/jupyter-edu-book/}}.
   \item Administració centralitzada dels recursos (espai de disc, programes...) i dels permissos
         disponibles per a les persones usuàries.
   \item Possibilitat de compartir bases de dades grans localitzades al mateix servidor.
   \item L'entorn d'usuari és permanent\footnote{A la reunió ho vaig descriure com una sèrie de màquines virtuals creades cada vegada que la persona usuària es connecta; però no és això. Ho havia confòs amb el \href{https://jupyter.org/binder}{BinderHub}}.
   \end{enumerate}

Actualment, Anaconda Cloud ofereix un JupyterHub gratuït, amb recursos limitats i sota
registre, així com altres modalitats de pagament: \href{https://anaconda.cloud/anaconda-tools}{https://anaconda.cloud/anaconda-tools}.
Pot servir per fer-se una idea del que podria oferir la Universitat de València.

   \item[Servidor de git] Un servidor de \textsf{git} pot ser tan senzill que només
s'utilitze a través de la línia de comandaments o pot tenir una interfície gràfica en la web,
a la manera de GitHub. Una opció amb interfície web i amb llicència oberta MIT Expat seria
la versió \emph{Community Edition} (CE) de \href{https://gitlab.com/gitlab-org/gitlab-foss/-/blob/master/README.md}{GitLab}. Els beneficis d'un servidor de git per a la docència poden no
ser obvis, i s'entén que aquest servei no seria prioritari. Tanmateix, crec que val la pena
considerar-ho. La idea és que els materials didàctics, que any rere any compartim entre
nosaltres, estarien molt ben allotjats en repositoris, a la manera del programari lliure,
on qualsevol puga descarregar-se'n una còpia i contribuir a la seua actualització. Almenys
entre el professorat disposat a substituir els PowerPoints per ioslides o documents de \LaTeX,
açò seria de gran ajuda.

   \item[Overleaf] Es va comentar que algun altre centre (Matemàtiques?) havia sol·licitat
algun servei semblant a Overleaf, per compartir documents de \LaTeX. És una idea excel·lent.
Però no va quedar clar si el que se sol·licita és només un compte institucional en una plataforma
existent o una implementació pròpia. Jo no conec una distribució lliure d'Overleaf o similar,
però és possible que n'existisquen. L'ús principal entenc que seria la col·laboració en l'escriptura
d'articles científics, però potser també l'elaboració de treballs de grup per a classe
per part d'estudiants, o la redacció d'actes de reunions, etc.
\end{description}

\section{Compromissos}
\begin{enumerate}
\item Per part del Servei d'Informàtica, afegir als ordinadors de les aules màquines virtuals 
amb un sistema operatiu Linux, on puguem sol·licitar que s'instal·le programari.
\item Per part de la Facultat de Ciències Biològiques, elaborar la consulta sobre la
necessitat dels serveis informàtics esmentats i reprendre'n la sol·licitud, si escau,
més endavant, amb un major suport.
\end{enumerate}

\end{document}
