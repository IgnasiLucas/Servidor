\documentclass[a4paper,12pt]{article}
\usepackage[T1]{fontenc}
\usepackage[utf8]{inputenc}
\usepackage[catalan]{babel}
\usepackage{booktabs}

\usepackage[colorlinks=true]{hyperref}
\title{Sol·licitud de serveis informàtics per la Facultat de Ciències Biològiques}
\begin{document}
\maketitle
\section{Introducció}
Actualment la Facultat de Ciències Biològiques compta amb sis aules d'informàtica
a la planta baixa de l'edifici B. Les aules estan equipades amb ordinadors que
s'inicien en un entorn de Linux i poden executar aplicacions de MS Windows en un
entorn virtual. Abans de l'inici de cada quadrimestre, el professorat sol·licita
a les persones administradores quines aplicacions necessita instal·lades a l'entorn
de MS Windows.

Aquesta manera d'oferir recursos informàtics a l'alumnat ha funcionat adequadament
durant molts anys en què el paradigma d'\textbf{alfabetització informàtica} era el
de poder utilitzar individualment programes informàtics privatius amb una interfície
gràfica d'usuària. Amb l'expansió del programari lliure, la popularització d'alguns
llenguatges de programació i l'adopció de tecnologies Web 2.0, les necessitats docents
i les expectatives dels i de les estudiants han canviat. Als paràgrafs següents
argumentem que el sistema de gestió de les aules d'informàtica ha quedat obsolet i
necessita ser complementat, si no substituït, per recursos informàtics diferents.

\section{Limitacions de les aules d'informàtica}
\subsection{Incapacitat d'administració del programari}
Només podem sol·licitar la instal·lació o l'actualització
de programari als ordinadors de les aules d'informàtica en dos moments a l'any: abans de
l'inici de cada quadrimestre. Aquesta rigidesa té conseqüències negatives:

\begin{enumerate}
\item Els i les estudiants perden l'oportunitat d'aprendre a instal·lar el programari.
\item Perdem l'oportunitat d'utilitzar versions més recents dels programes amb un cicle
ràpid d'actualitzacions.
\item No podem variar el programa d'activitats ni improvisar-ne de noves, si no és
utilitzant els mateixos programes que ja hi ha instal·lats.
\end{enumerate}

\subsection{Disponibilitat estrictament local del programari}
S'entén que un control rígid del programari instal·lat 
garanteix el respecte de llicències d'ús privatives. Siga com siga, cada vegada és més
freqüent que els i les estudiants porten a l'aula d'informàtica els seus propis ordinadors
portàtils, amb l'esperança de poder realitzar la pràctica en el seu propi ordinador.
No hi ha dubte dels enormes beneficis didàctics de
treballar amb l'ordinador propi, tant des de l'aula com des de casa. Actualment,
aquesta possibilitat és terriblement incòmoda i frustrant, perquè els ordinadors
particulars porten una diversitat de versions de diferents sistemes operatius, on la
instal·lació del programari necessari pot incórrer en problemes de llicència o
d'incompatibilitats. Quan, a pesar de tot, s'aconsegueix realitzar la pràctica amb
l'ordinador propi, ens trobem amb la paradoxal situació d'estar infrautilitzant els
recursos de la Universitat. En definitiva, les aules ofereixen un entorn homogeni i
convenient, però al preu de no poder-se utilitzar més que allà mateix. Aquesta obvietat
resulta ser un hàndicap important de cara a l'aprenentatge, i fa que l'opció cada vegada
més freqüent d'utilitzar ordinadors propis siga preferible.

\subsection{Obstrucció de l'ús de l'entorn Linux}
Alguns programes instal·lats a l'entorn MS Windows podrien ser executats igualment en
l'entron Linux. Aquesta possibilitat és molt desitjable per tots els motius que fan
el programari lliure preferible al privatiu.

Altres programes senzillament només existeixen per a Linux. En aquest cas, no tenim
opció d'utilitzar-los a les aules d'informàtica, tot i existir els recursos necessaris.
Fins ara ha sigut impossible sol·licitar la instal·lació
de programari docent en l'entorn Linux de les aules d'informàtica. Els administradors
de les aules d'informàtica només atenien les peticions d'instal·lació de programari en
MS Windows.

Certament, la persona usuària sempre té la possibilitat d'instal·lar programari dins la seua
pròpia carpeta, on gaudeix de permissos d'escriptura. Per exemple, no és impossible
instal·lar el gestor de paquets \textsf{conda}, i amb ell instal·lar qualsevol altre programa en
una carpeta local. Tanmateix, el procés pot prendre molts minuts de classe, i s'hauria
de repetir a cada vegada que es fa servir un ordinador diferent de l'aula.

Considerem que a les aules d'informàtica de la Facultat de Biologia és necessari poder
treballar i ensenyar en un entorn Linux.

\subsection{Alienació de l'estudiant}
En enfrontar-se a un sistema operatiu hermètic, sobre el qual la persona usuària no té
cap control, l'actitud esdevé la d'un client cap a un proveïdor, no la d'un estudiant
universitari. Des d'un punt de vista didàctic, l'entorn MS Windows i totes les aplicacions
amb interfície gràfica d'usuària són nefastes pels motius següents:

\begin{enumerate}
\item Cancel·len la curiositat, en tant que el programari funciona com una caixa negra,
que respon als \emph{clicks} sense revelar-ne el com. Com per art de màgia.
\item Afegeix una càrrega cognitiva completament inútil: aprendre l'ús d'una interfície
gràfica concreta (on són els menús, quins botons fan què, com es configura, etc.) no té
cap altra aplicació en absolut.
\item Les interfícies gràfiques d'usuària són llenguatges analògics, no generatius. Igual
que si ens comunicàrem per gestos, el que podem comunicar a un ordinador amb una interfície
gràfica és extremadament limitat.
\item Manipular un ordinador amb interfícies gràfiques generalment implica renunciar a
la reproduïbilitat: no guardem registre d'on i quan hem fet \emph{click}. I per tant,
reproduïr un resultat o explicar com l'hem obtingut esdevé sovint una tasca impossible.
Res més lluny del que hauria de ser l'ensenyança universitària.
\end{enumerate}

\section{Proposta de serveis informàtics}
La proposta gira entorn a un canvi de paradigma, en què el protagonisme passa del
programari a la persona usuària. En primer lloc, cal posar en valor els \textbf{llenguatges
de programació interpretats} com a mitjà (més que com a fi) de qualsevol aprenentatge. Els
llenguatges de programació interpretats es caracteritzen
per la possibilitat d'especificar les ordres en un format de text pla, que pot ser
interpretat i executat en qualsevol maquinari. El fet de ser un \textbf{estàndard}
universal, fa del text pla un vehicle fantàstic de tota classe d'instruccions i comunicacions.
Quan ens comuniquem amb els ordinadors a través de text pla, és a dir, a través d'una línia
de comandaments o d'un \emph{script}, tenim tots els avantatges que les interfícies gràfiques
ens neguen:

\begin{enumerate}
\item Les ordres o comandaments són transparents: generen un resultat i alhora mostren
com ho fan. No hi ha, per tant, barreres a la curiositat.
\item La càrrega cognitiva d'aprendre a programar és certament molt major que la d'aprendre a fer
\emph{clicks}, però no és inútil en absolut, sinó que fins i tot un coneixement molt
superficial pot capacitar per fer moltes coses, en qualsevol assignatura i en la
vida professional.
\item Un llenguatge de programació és difícil d'aprendre perquè la relació entre el
significat i el significant és arbitrària, exactament igual que en un llenguatge natural. I això és,
precisament, el que fa que el llenguatge siga \emph{generatiu}: les possibilitats de creació
de significats són infinites. La programació és una habilitat sobre la qual es pot \emph{construir},
mentre que l'ús d'una interfície gràfica, no.
\item Les odres especificades en text pla són fàcilment guardades en documents o \emph{scripts},
que registren i permeten documentar tot allò que li demanem a l'ordinador que faça. És a dir,
l'ús d'un llenguatge de programació pràcticament ens \emph{obliga} a registrar la manera en què
hem obtingut un resultat i ens facilita enormement compartir i col·laborar amb altres en
l'obtenció de resultats, qualsevol que siga la naturalesa de la tasca.
\end{enumerate}

Amb l'objectiu de beneficiar la comunitat universitària de tots aquests avantatges, 
a continuació proposem la implementació de nous serveis informàtics.

\subsection{Facilitar l'ús de l'entorn Linux}
\subsubsection{De manera local}
En primer lloc, el canvi més immediat que es pot implementar per començar a sol·lucionar el
problema és oferir la possibilitat d'instal·lar programari en l'entorn de Linux. Si més no,
de moment, en les mateixes finestres temporals en què se'ns permet sol·licitar la instal·lació
de programari en MS Windows.

Gran part del problema se solucionaria si a l'entorn Linux tiguérem instal·lada alguna cosa
més que l'estrictament necessari. Algunes eines de programari lliure són d'aplicació molt
general en qualsevol assignatura, com per exemple: \textsf{git}, \LaTeX, \textsf{R}, Python
i conda, per esmentar-ne uns pocs.

\subsubsection{De manera remota}
Idealment, els i les estudiants haurien de poder accedir remotament, per \textsf{ssh},
a un servidor Linux. Fins i tot si l'accés es realitzara exclusivament a través del
terminal, sense entorn gràfic, aquesta experiència seria molt valuosa per a l'aprenentatge.
Si cada persona usuària tinguera el seu directori propi i els canvis foren permanents,
el servidor podria utilitzar-se a manera de núvol. A més de poder accedir des de casa com
des de l'aula d'informàtica, els i les estudiants tindrien facilitat per accedir a recursos comuns
del servidor, com ara dades amb les quals realitzar exercicis pràctics.

Ens consta que alguns centres de la Universitat de València disposen ja d'un servidor de
càlcul per a finalitats docents. Actualment les ciències biològiques tenen tanta necessitat
de computació com qualsevol altra disciplina, o més. Per tant, sol·licitem la dedicació
d'un servidor de càlcul a tasques docents de la facultat.

\subsection{RStudio-server}
L'ús d'\emph{scripts} en la docència amb ordinadors és fonamental. Els \emph{scripts}
són documents vius, que poden ser editats durant una pràctica i poden convertir-se també
en els apunts o annotacions amb què els i les estudiants conserven allò  que han aprés.
Actualment existeix una varietat de formats capaços de combinar en un mateix document
les instruccions que donem a l'ordinador, els resultats d'aquestes instruccions si són
imprimibles (figures, taules o resultats numèrics), i també les nostres anotacions o
comentaris que documenten o expliquen l'anàlisi i els resultats. Una de les modalitats
són els documents Rmarkdown, propis del llenguatge \textsf{R} però capaços d'interpretar
ordres de Python, Bash, etc. Els documents Rmarkdown de text pla generen documents
en format PDF o HTML en ser compilats.

El programa RStudio té una versió "servidor", RStudio-server, que permet l'ús compartit
del servidor de càlcul a través d'una interfície típica de RStudio en una pestanya del
navegador. Aquesta possibilitat no només facilita l'ús i l'edició de documents Rmarkdown,
sinó que permetria a tots els membres d'un grup executar les anàlisis o els seus \emph{scripts}
en un mateix ambient, amb els mateixos paquets instal·lats i amb accés a les mateixes dades.

\subsection{JupyterHub}
Una altra modalitat de documents executables són els \emph{Jupyter notebooks}
("quaderns Jupyter"\ en avant), que s'editen
i es visualitzen a través del navegador, a manera d'una aplicació web, i tenen l'avantatge
de ser més dinàmics: pots actualitzar seccions concretes del document sense haver de compilar-lo
tot sencer. Els quaderns Jupyter accepten instruccions en multitud de llenguatges de
programació, incloent Python, Julia, \textsf{R}, Bash, etc. A l'enllaç següent, amb una
mica de paciència, s'activa un JupyterLab amb un parell de quaderns Jupyter d'exemple.

\href{https://mybinder.org/v2/gh/IgnasiLucas/Khi2/soca}{https://mybinder.org/v2/gh/IgnasiLucas/Khi2/soca}

Aquests documents tenen potencials didàctics enormes \cite{Barba2019}. Un dels seus avantages
és la facilitat amb què poden ser compartits. En estar basats en programari lliure i formats
estàndard, són documents dissenyats per la difussió i la col·laboració. Actualment existeixen
solucions de programari lliure per oferir de forma remota accés als entorns de computació
virtuals on resideixen i poden ser actualitzats i executats aquests documents. El programari
que fa possible compartir un conjunt de quaderns Jupyter amb una classe o un grup
de persones col·laboradores és \href{https://jupyter.org/hub}{JupyterHub} \cite{JupyterHub}.

Si la Universitat de València tinguera instal·lat i en execució un JupyterHub, qualsevol
membre de la comunitat universitàira podria connectar-se, des de qualsevol ordinador amb
internet, i executar els seus treballs en entorns comuns, prèviament definits pel
professorat.

Existeixen solucions comercials, com \href{https://anaconda.cloud/}{Anaconda Cloud}. Però
es tracta de programari lliure, i si la Universitat de València gestionara el seu propi
JupyterHub, estudiants i professorat podríem utilitzar els nostres comptes de la Universitat
per accedir al servei, sense haver d'acceptar els termes i condicions d'una empresa, ni
patir les limitacions pròpies dels comptes gratuïts dels serveis comercials.

\subsection{Servidor de \textsf{git}}
El programa \textsf{git} és un sistema de control de versions. S'utilitza intensament en
l'àmbit del desenvolupament de programari. Serveix principalment per mantenir fora de la
vista però ben guardades les versions anteriors d'un conjunt de documents, en una carpeta
de l'ordinador. Això ajuda enormement a tenir els arxius propis ben ordenats i evitar
situacions com aquella popularitzada en la tira còmica
\href{https://phdcomics.com/comics/archive.php?comicid=1531}{"Final.doc"}. A més, \textsf{git}
guarda registre de qui va introduir quins canvis i quan, la qual cosa facilita molt la
col·laboració o l'edició conjunta de documents.

Tot i haver estat concebut per ajudar al desenvolupament de programari, \textsf{git}
té aplicació en qualsevol conjunt de documents que han de ser editats al llarg del temps,
per una o més persones. En l'àmbit universitari, aquests documents poden ser: un treball de
classe qualsevol, una tesi, o uns materials didàctics. Tot i que la corba d'aprenentatge no és
suau, un coneixement bàsic de \textsf{git} permet gaudir de la major part dels seus beneficis.

Tot i que \textsf{git} pot executar-se localment en l'ordinador particular, gran part de la
seua utilitat s'aprecia en els treballs col·laboratius. En aquest cas, és necessari un
servidor de \textsf{git} remot, al qual totes les persones participants d'un projecte han
de poder connectar-se, be per descarregar les actualitzacions o bé per proposar-ne de noves.
El mateix programa ofereix mecanismes per evitar i gestionar els possibles conflictes entre
versions diferents d'un mateix document. L'exemple més popular de servidor de \textsf{git}
és la plataforma \href{https://github.com}{GitHub}, que ofereix plans gratuïts per compartir
carpetes de forma pública. A l'enllaç següent es pot accedir a una carpeta en GitHub que
conté materials didàctics:

\href{https://github.com/IgnasiLucas/Bioinformatica\_33190}{https://github.com/IgnasiLucas/Bioinformatica\_33190}

La Universitat de València hauria d'oferir el seu propi servidor de \textsf{git}, igual que
ofereix el servei de núvol o de correu electrònic. Moltes altres universitats ho fan. D'aquesta
manera, les carpetes podrien ser compartides més fàcilment només dins l'àmbit de la universitat,
i sense haver de recórrer a serveis externs.

\section{Conclusions}
Les aules d'informàtica no cobreixen les necessitats d'aprenentatge de les generacions actuals
d'estudiants. En qualsevol carrera professional és cada vegada més habitual haver de tenir nocions
de programació. Tal com passa amb els llenguatges naturals, els llenguatges de programació
s'aprenen millor en el context del seu ús, que com a objectiu de l'aprenentatge. Si a la
Facultat de Ciències Biològiques disposem de les infraestructures sol·licitades, serà més
fàcil que diferents assignatures es coordinen per utilitzar el mateix llenguatge de programació
a les pràctiques amb ordinador. Així, sense haver d'impartir més que les nocions bàsiques,
i sense pretendre que les persones graduades esdevinguen també expertes programadores, haver-los
exposat sistemàticament a un llenguatge de programació repercutirà molt positivament en la seua
preparació:

\begin{enumerate}
\item La familiaritat mínima amb una línia de comandaments desmitifica la tecnologia,
proporciona seguretat i possibilita l'aprenentatge posterior.
\item Havent entés com funciona un llenguatge de programació és molt més fàcil aprendre'n un altre.
És a dir, allò essencial no és triar el llenguatge del futur, sinó fer-los treballar amb un dels
llenguatges disponibles.
\item L'edició i la gestió d'\emph{scripts}, així com l'ús de \textsf{git} promou els bons
hàbits de treball amb ordinador, essencials per preservar la integritat de les dades, la
reproduïbilitat dels resultats i la traçabilitat dels canvis i de l'autoria dels documents.
\end{enumerate}

En definitiva, demanem actualitzar el concepte d'alfabetització informàtica per incloure nocions
de programació. De la mateixa manera que tots aprenem a llegir i escriure, sense necessitat
d'arribar a ser escriptors o escriptores professionals, igualment tots ens podem beneficiar
de saber programar, sense necessitat d'arribar a ser enginyers o enginyeres informàtiques
\cite{Vee2017}.

Per poder aconseguir-ho, necessitem que la Universitat actualitze els seus recursos informàtics.
Ja no n'hi ha prou amb una aula d'informàtica que dóna accés a uns pocs programes privatius
en un lloc i unes hores concretes. Aquest model caducat seria comparable a tenir un museu de
llibres en lloc d'una biblioteca.

\bibliography{sollicitud}
\bibliographystyle{ieeetr}
\end{document}

