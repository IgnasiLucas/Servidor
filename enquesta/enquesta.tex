\documentclass[a4paper,12pt]{article}
\usepackage[utf8]{inputenc}
\usepackage[T1]{fontenc}
\usepackage[catalan]{babel}
\usepackage{booktabs}
\usepackage[colorlinks=true]{hyperref}
\title{Enquesta sobre serveis informàtics generals per a la docència}
\hyphenation{Python}

\newcounter{preg}[section]
\renewcommand{\theenumi}{\arabic{section}.\arabic{preg}}

\begin{document}
\maketitle
Un aspecte important de la transformació digital de la societat actual és l'ús creixent
dels llenguatges de programació, tant en àmbits professionals com educatius, a tots els
nivells i en qualsevol àrea. En la docència universitària, llenguatges com Python o R
estan passant de ser objectius d'aprenentatge en cursos especialitzats a ser un recurs
didàctic o vehicle d'ensenyança de qualsevol matèria i en qualsevol grau. Part de la
popularitat d'aquests llenguatges
de programació és conseqüència de poder-se expressar en text pla, un estàndard
universal fàcilment interpretable en qualsevol sistema operatiu. Aquesta
transparència dels materials didàctics elaborats en Python, R o gairebé qualsevol altre
llenguatge de programació facilita enormement que siguen compartits, reutilitzats i
actualitzats de manera col·lectiva i amb llicències obertes, tant per una comunitat docent
com per les i els estudiants. Altres tecnologies basades en el text pla (\LaTeX, git,
markdown, Jupyter notebooks, etc.) contribueixen a donar un caràcter obert a l'educació superior
i a desenvolupar les habilitats computacionals bàsiques de les i els estudiants.

Amb l'única finalitat de planificar la possible implementació de nous serveis informàtics que
faciliten l'adopció d'aquestes tecnologies, el Servei d'Informàtica necessita conéixer el
grau d'interés que susciten. Per aquest motiu, us demanem que contesteu les preguntes següents,
que no us ocuparan més de 5 minuts. Les respostes individuals seran completament anònimes i no seran
compartides amb tercers. Els estadístics de resum seran publicats a la pàgina web del Servei
d'Informàtica.

\section{Dades personals}
\setcounter{preg}{1}

\begin{enumerate}
\item \stepcounter{preg} Centre al qual pertany:
\item \stepcounter{preg} Categoria professional:
\item \stepcounter{preg} Gènere:
\item \stepcounter{preg} Edat:
   \begin{enumerate}
   \item Menys de 31.
   \item 31-40.
   \item 41-50.
   \item 51-60.
   \item Més de 60.
   \end{enumerate}
\end{enumerate}

\section{Participació en la docència de pràctiques amb ordinador}
\begin{enumerate}
\stepcounter{preg}
\item \stepcounter{preg} Participes en la docència de pràctiques amb ordinador?
   \begin{itemize}
   \item Sí.
   \item No.
   \item Ara no, però ho he fet o podria fer-ho.
   \end{itemize}

\item \stepcounter{preg} Els programes que s'utilitzen en aquestes pràctiques, són programari lliure?
   \begin{itemize}
   \item Sí.
   \item No.
   \end{itemize}

\item \stepcounter{preg} Els i les estudiants tenen facilitat per instal·lar aquests programes als seus ordinadors?
   \begin{itemize}
   \item Sí.
   \item No.
   \item Depén del sistema operatiu.
   \end{itemize}

\item \stepcounter{preg} Podries dir quins són els principals programes que utilitzeu en les pràctiques amb ordinador?
\vspace*{1cm}

\item \stepcounter{preg} Les aules d'informàtica disposen d'aquest programari instal·lat?
   \begin{itemize}
   \item Sí, a MS Windows.
   \item Sí, a Linux.
   \item No.
   \end{itemize}

\item Comenta breument quines incidències has tingut amb aquests programes, si n'ha hagut cap.
\vspace*{1cm}
\end{enumerate}

\section{Llenguatges de programació}
\stepcounter{preg}
\begin{enumerate}
\item \stepcounter{preg} Tens coneixement d'algun llenguatge de programació?
   \begin{itemize}
   \item Sí.
   \item No.
   \end{itemize}

\item Quins?
   \begin{itemize}
   \item Python.
   \item R.
   \item Julia.
   \item Go.
   \item C/C++.
   \item Perl.
   \item Pascal.
   \item Bash.
   \item Awk.
   \item Altres:
   \end{itemize}
\end{enumerate}

\section{Sobre RStudio}
\stepcounter{preg}
\begin{enumerate}
\item \stepcounter{preg} Coneixes i utilitzes l'entorn RStudio? \cite{RStudio}
   \begin{itemize}
   \item Sí, el conec, però no l'utilitze.
   \item Sí, l'utilitze, però no per la docència.
   \item Sí, l'utilitze fins i tot per la docència.
   \item No, ni el conec ni l'utilitze.
   \end{itemize}

\item \stepcounter{preg} Si ho fas, coneixes i utilitzes els documents dinàmics de tipus Rmarkdown?
   \begin{itemize}
   \item Si, els conec, però no els utilitze.
   \item Sí, els utilitze, però no en la docència.
   \item Sí, els utilitze fins i tot per la docència.
   \item No, ni els conec ni els utilitze.
   \end{itemize}

\item Creus que la UV hauria d'oferir un servidor d'RStudio per al seu ús remot en la
      docència, per part d'estudiants i del professorat?
   \begin{itemize}
   \item Sí.
   \item No.
   \item Indiferent.
   \end{itemize}
\end{enumerate}

\section{Jupyter notebooks}
\stepcounter{preg}
\begin{enumerate}
\item \stepcounter{preg} Coneixes i utilitzes l'entorn Jupyter (Jupyter notebooks o Jupyter lab)? \cite{Barba2019}
   \begin{itemize}
   \item Sí, el conec, però no l'utilitze.
   \item Sí, l'utilitze, encara que no per la docència.
   \item Sí, l'utilitze fins i tot per la docència.
   \item No, ni el conec ni l'utilitze.
   \end{itemize}

\item Creus que la UV hauria d'oferir un servidor Jupyter Hub per a ús remot en la docència,
      per part d'estudiants i del professorat? \cite{JupyterHub}
   \begin{itemize}
   \item Sí.
   \item No.
   \item Indiferent.
   \end{itemize}
\end{enumerate}

\section{Sobre l'ús de llenguatges interpretats en les pràctiques amb ordinador}
\stepcounter{preg}
\begin{enumerate}
\item \stepcounter{preg} Quins avantatges creus que tindria substituir una aplicació d'interfície gràfica per un
      \emph{script} (document de text pla executable) de llicència oberta en la docència de
      pràctiques amb ordinador?
   \begin{itemize}
   \item Reproduïbilitat.
   \item Portabilitat a qualsevol sistema operatiu.
   \item Aprenentatge passiu d'un llenguatge de programació.
   \item Eliminació de \emph{caixes negres} en l'ensenyança.
   \item Llibertat per millorar o adaptar el codi a cada pràctica concreta.
   \item Altres:
   \item No cap.
   \end{itemize}

\item \stepcounter{preg} Quins desavantatges o dificultats comportaria substituir una aplicació d'interfície gràfica
      per un \emph{script} de llicència oberta en la docència de pràctiques amb ordinador?
   \begin{itemize}
   \item La necessitat d'aprendre el llenguatge de programació per part del professorat.
   \item L'esforç d'haver de preparar la pràctica de nou.
   \item L'ús d'una interfície menys atractiva per les i els estudiants.
   \item Altres:
   \item No cap.
   \end{itemize}

\item \stepcounter{preg} T'agradaria rebre informació sobre com aplicar els llenguatges de programació a la docència?
   \begin{itemize}
   \item Sí.
   \item No.
   \end{itemize}

\item Si hi ha cap llenguatge de programació que t'agradaria aprendre, quin és?
\vspace*{1cm}
\end{enumerate}

\section{Sobre els sistemes de control de versions}
\stepcounter{preg}
\begin{enumerate}
\item \stepcounter{preg} Coneixes cap sistema de control de versions (e.g. \textsf{git}, Subversion, Mercurial) i
      per a què serveixen?
   \begin{itemize}
   \item Sí, però no els faig servir.
   \item Sí, en faig servir algun, però no en relació amb la docència.
   \item Sí, en faig servir algun, fins i tot amb documents destinats a la docència.
   \item No.
   \end{itemize}

\item \stepcounter{preg} Éssent els materials docents documents generalment compartits i actualitzats amb certa
      freqüència, per quins motius creus que no solen comptar amb un sistema de control de
      versions?
   \begin{itemize}
   \item En realitat, només es comparteixen ocasionalment i no necessitem harmonitzar les versions.
   \item Per desconeixement dels sistemes de control de versions per part del professorat.
   \item Perquè la UV no ofereix cap repositori remot on mantenir i compartir els materials
         docents.
   \item Perquè els materials docents inclouen imatges o fragments protegits amb llicències
         privatives que no poden ser compartits obertament.
   \item Perquè generalment utilitzem formats privatius (MS Word, .ppt, etc.), que no poden
         editar-se en qualsevol sistema operatiu.
   \item Altres:
   \vspace*{1cm}
   \end{itemize}

\item \stepcounter{preg} Creus que mantenir col·lectivament els materials docents sota un sistema de control de
      versions seria desitjable?
   \begin{itemize}
   \item Sí.
   \item No.
   \item Indiferent.
   \end{itemize}

\item \stepcounter{preg} Estaries disposat o disposada a rebre formació sobre l'ús d'un sistema de control de versions
      com \textsf{git}?
   \begin{itemize}
   \item Sí.
   \item No.
   \end{itemize}

\item \stepcounter{preg} Creus que la UV hauria d'oferir el seu propi  servidor de repositoris de \textsf{git}, com
      un Gitlab? \cite{Paderborn2023,PereiraBraga2023}
   \begin{itemize}
   \item Sí.
   \item No.
   \item Indiferent.
   \end{itemize}
\end{enumerate}

\section{Sobre l'ús d'un servidor de càlcul per la docència}
\stepcounter{preg}
\begin{enumerate}
\item \stepcounter{preg} En algunes assignatures s'utilitzen o podrien utilitzar-se bases de dades grans i/o una intensitat
      de computació elevada, que són difícils d'implementar als ordinadors d'una aula d'informàtica.
      En la teua docència habitual, trobes cap motiu com aquests per a què estudiants i professorat
      hagueren de connectar-se a un servidor de càlcul remot?
   \begin{itemize}
   \item Sí.
   \item No.
   \end{itemize}

\item Si és així, podries concretar quin recurs seria més accessible a través d'un servidor de
      càlcul remot?
\vspace*{1cm}
\end{enumerate}

\bibliography{../references}
\bibliographystyle{ieeetr}
\end{document}
