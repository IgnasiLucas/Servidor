\documentclass[a4paper,12pt]{article}
\usepackage[utf8]{inputenc}
\usepackage[T1]{fontenc}
\usepackage[catalan]{babel}
\usepackage{booktabs}
\usepackage[colorlinks=true]{hyperref}
\title{Enquesta sobre serveis informàtics generals per a la docència}
\hyphenation{Python}

\begin{document}
\maketitle
A la Facultat de Ciències Biològiques, cada vegada és més habitual que en alguna
assignatura s'utilitze un llenguatge de programació, com R o Python,
per realitzar exercicis o pràctiques amb ordinador. Aquestes iniciatives individuals
estan sovint arrelades en una concepció oberta de l'educació i de la ciència \cite{Schlagwein2017},
i contribueixen a proporcionar habilitats computacionals bàsiques a estudiants de
qualsevol grau \cite{Vee2017}. Tanmateix, l'adopció de llenguatges de programació en
les nostres aules es troba amb dificultats i necessita d'un recolzament institucional
per part de la Universitat.

L'objectiu d'aquesta enquesta és compartir aquesta percepció amb el professorat de tots
els centres de la Universitat de València (UV en avant) i arreplegar el suport necessari per sol·licitar
la posada en marxa de serveis informàtics generals que faciliten l'adopció de llenguatges de
programació en l'ensenyança de qualsevol matèria i contribuïsquen a l'educació oberta.

Les dades arreplegades són completament anònimes i no seran compartides amb tercers. Només
seran publicats els estadístics de resum, en un informe que podreu consultar al web...
Si alguna persona està interessada en ajudar a coordinar una sol·licitud inter-facultativa
dels serveis informàtics generals proposats, pot posar-se en contacte amb...

\begin{enumerate}
\item Centre al qual pertany:
\item Gènere:
\item Edat:
   \begin{enumerate}
   \item Menys de 30.
   \item 30-40.
   \item 40-50.
   \item 50-60.
   \item Més de 60.
   \end{enumerate}
\end{enumerate}

Marca el grau de conformitat amb cadascuna de les afirmacions següents, entre 
zero (en complet desacord) i cinc (completament d'acord).

{\small
\begin{tabular}{|r|p{9cm}|c|c|c|c|c|c|}
\toprule
&&0&1&2&3&4&5\\
\midrule
4&Tinc bons coneixements d'almenys un llenguatge de programació.&&&&&&\\
\midrule
5&L'ensenyança de les matèries en què participe podria millorar amb l'adopció d'un llenguatge de programació.&&&&&&\\
\midrule
6&Participe o podria participar activament en la docència de pràctiques amb ordinador.&&&&&&\\
\midrule
7&Estaria disposat o disposada a rebre formació en l'aplicació de la programació informàtica a la docència.&&&&&&\\
\midrule
8&Les aules d'informàtica amb què comptem disposen de Linux i de tots els programes que necessitem.&&&&&&\\
\midrule
9&La major part de programes informàtics que ensenyem a utilitzar són programari lliure.&&&&&&\\
\midrule
10&Els i les estudiants poden instal·lar fàcilment als seus ordinadors els programes informàtics que utilitzem a l'aula.&&&&&&\\
\midrule
11&Conec i utilitze amb certa freqüència documents dinàmics d'Rmarkdown en l'entorn d'RStudio \cite{RStudio}.&&&&&&\\
\midrule
12&Conec i utilitze amb certa freqüència els quaderns Jupyter (Jupyter Notebooks) \cite{Barba2019}.&&&&&&\\
\midrule
13&La UV hauria d'oferir un servidor d'RStudio perquè estudiants i professorat puguen connectar-se remotament.&&&&&&\\
\midrule
14&La UV hauria d'oferir un servidor JupyterHub de quaderns Jupyter \cite{JupyterHub}.&&&&&&\\
\midrule
15&La UV hauria d'oferir accés per \textsf{ssh} a un servidor de càlcul per a estudiants.&&&&&&\\
\midrule
16&La UV hauria d'allotjar el seu propi servidor de repositoris de \textsf{git} (e.g., Gitlab) per a professorat i estudiants \cite{Paderborn2023,PereiraBraga2023}.&&&&&&\\
\midrule
17&La UV hauria de mantenir un node propi d'una xarxa social federada, com Mastodon, i oferir comptes a tota la comunitat, igual que amb el correu electrònic \cite{Brembs2023,mastodon}.&&&&&&\\
\bottomrule
\end{tabular}
}

Si vols suggerir algun altre servei o fer algun comentari, pots fer-ho ací:
\vspace*{3cm}

\bibliography{../references}
\bibliographystyle{ieeetr}
\end{document}

