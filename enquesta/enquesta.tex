\documentclass[a4paper,12pt]{article}
\usepackage[utf8]{inputenc}
\usepackage[T1]{fontenc}
\usepackage[spanish,catalan]{babel}
\usepackage{booktabs}
\usepackage[colorlinks=true]{hyperref}
\title{Enquesta sobre serveis informàtics generals i per a la docència}

\begin{document}
\section{Introducció}
En el context de la progressiva transformació digital de la societat,
l'Equip Deganal de la Facultat de Ciències Biològiques ha identificat
algunes necessitats digitals en la docència que la Universitat de València
encara no cobreix. Amb el propòsit de compartir i elaborar una proposta
comuna que puga beneficiar a tots els centres de la Universitat, volem
saber, mitjançant aquesta enquesta, si la comunitat docent comparteix
en general la percepció d'aquestes o d'altres necessitats que pogueren
ser adreçades des del Servei d'Informàtica.

\section{Dades personals}
\begin{enumerate}
\item Centre al qual pertany:
\item Gènere:
\item Edat:
\item Cos professional:
\end{enumerate}

\section{Servidor de càlcul per a docència}
\begin{enumerate}
\item A les matèries en què imparteixes docència, feu servir aula d'informàtica? (Si la resposta
és negativa, passa a la secció següent.)
   \begin{enumerate}
   \item Sí.
   \item No.
   \end{enumerate}
\item A l'aula d'informàtica, podries executar programes docents en Linux, si fóra necessari?
   \begin{enumerate}
   \item Sí.
   \item No.
   \end{enumerate}
\item Fas servir algun llenguatge de programació? En cas afirmatiu, quin?
   \begin{enumerate}
   \item Sí.
   \item No.
   \end{enumerate}
\item Algunes veus, com la de \href{https://youtu.be/mGc6clf_Wt4}{Mark Guzdial} advoquen per l'ús dels
llenguatges de programació en la docència de qualsevol altra matèria, com a eina per manipular i
comprendre el món que ens envolta. Sobre el paper dels llenguatges de programació en la docència,
amb quina afirmació t'identifiques més? (Pots marcar-ne més d'una).
   \begin{enumerate}
   \item A les matèries que impartisc, la programació informàtica no té cap aplicació.
   \item Crec que seria útil, però no tinc temps de preparar-ho.
   \item Estaria disposat o disposada a rebre formació per introduir algun llenguatge de programació en les meues classes.
   \item Ocasionalment utilitze algun llenguatge de programació com a mitjà per ensenyar la meua matèria.
   \item Utilitze amb freqüència algun llenguatge de programació per exemplificar o demostrar conceptes.
   \item Altra:
   \end{enumerate}

\item En molts àmbits professionals i acadèmics, el programari lliure ha guanyat popularitat
en les últimes dècades. Quin ús fas del programari lliure a l'aula d'informàtica?
   \begin{enumerate}
   \item No concec programari lliure que puga satisfer les necessitats de la docència que impartisc.
   \item Preferisc utilitzar programari privatiu, per les garanties que ofereix i altres avantatges.
   \item Normalment utilitze programari lliure instal·lat a l'aula d'informàtica.
   \item Normalment utilitze programari lliure que els i les alumnes instal·len als seus ordinadors.
   \item Utilitzem programari lliure en un servidor remot (\emph{cloud computing}) del nostre centre.
   \item Altra:
   \end{enumerate}

\item Sobre l'ús de servidors de càlcul, o computació en el núvol, quina afirmació reflexa
millor la situació de la docència en la qual participes?
   \begin{enumerate}
   \item No utilitze ni tampoc necessite cap servidor de càlcul remot.
   \item No utilitze cap servidor de càlcul remot per a docència, però ho faria si estiguera disponible.
   \item Al nostre centre disposem d'un servidor de càlcul al qual els i les estudiants poden connectar-se remotament.
   \item En ocasions contractem serveis de computació al núvol per a finalitats docents.
   \item Altra:
   \end{enumerate}

\item L'ús creixent de llenguatges de programació en l'ensenyança va acompanyat per la
popularització de diferents entorns o plataformes que ho faciliten, com són 
\href{https://education.rstudio.com/teach/}{RStudio} i
\href{https://jupyter4edu.github.io/jupyter-edu-book/}{Jupyter Notebooks}.
Quin ús fas d'aquests programes en l'ensenyança? (Pots marcar més d'una opció).
   \begin{enumerate}
   \item No n'havia sentit parlar.
   \item N'he sentit parlar, però no els faig servir.
   \item Faig servir RStudio (o Jupyter Notebooks) en la meua investigació, però no en la docència.
   \item Conec el potencial d'aquestes eines en l'ensenyament i estaria disposada a rebre formació sobre el tema.
   \item Utilitze RStudio (o Jupyter Notebooks) a l'aula d'informàtica.
   \item Si la Universitat oferira un servidor de RStudio, probablement l'utilitzaria en la docència.
   \item Si la Universitat oferira un servidor de Jupyter Notebooks (JupyterHub), probablement l'utilitzaria en la docència.
   \item Altra:
   \end{enumerate}
\end{enumerate}

\section{Control de versions amb \textsf{git}}
El programa \href{https://git-scm.com/}{git} és el sistema de control de versions més popular
actualment, i s'utilitza des de fa temps en projectes col·laboratius d'elaboració de documents,
com per exemple (\href{https://doi.org/10.1111/2041-210X.14108}{però no sols}) en el desenvolupament de programari.
\begin{enumerate}
\item Quina experiència tens amb git o amb un altre sistema de control de versions? (Pots marcar més d'una opció.)
   \begin{enumerate}
   \item No els he utilitzat mai, i no em fan falta.
   \item No els he utilitzat mai, però tinc curiositat i m'agradaria rebre formació sobre el tema.
   \item He utilitzat control de versions al meu propi ordinador, però no en la docència.
   \item Utilitze control de versions en col·laboracions, però no en docència.
   \item Utilitze control de versions en l'elaboració de materials didàctics.
   \item Utilitze control de versions i repositoris remots (Github, Gitlab, etc.) en la interacció amb alumnes.
   \end{enumerate}

\item Quan vols compartir una carpeta amb materials didàctics, de quina manera ho fas? (Marca totes les que utilitzes).
   \begin{enumerate}
   \item A través del núvol de la Universitat de València.
   \item A través de l'Aula Virtual.
   \item A vegades, a través de serveis externs (Dropbox, etc.).
   \item A vegades, a través d'un servidor de git, com \href{https://github.com/}{Github}.
   \item Altres:
   \end{enumerate}

\item Si la Universitat oferira un servidor de \textsf{git} propi (Gitlab), el faries servir?
   \begin{enumerate}
   \item Probablement no, perquè no tinc la necessitat.
   \item Probablement no, perquè amb Github tinc prou.
   \item Probablement sí, sempre que rebera la formació adequada.
   \item Segurament sí, per poder triar amb qui compartir les carpetes.
   \item Segurament sí, perquè preferisc utilitzar serveis propis de la Universitat que serveis externs.
   \item Altra:
   \end{enumerate}
\end{enumerate}

\section{Altres serveis generals}
Recentment, \href{https://netzpolitik.org/2023/a-call-to-action-universities-of-the-world-into-the-fediverse/#!}{s'ha proposat}
que les universitats gestionen els seus propis servidors o nodes de les xarxes socials federades (Fedivers), com ara
\href{https://doi.org/10.1038/d41586-023-00486-3}{Mastodon}. Sobre l'ús acadèmic de les xarxes socials:
\begin{enumerate}
\item Per a la teua activitat docent i investigadora, participes en alguna xarxa social?
   \begin{enumerate}
   \item Sí.
   \item No.
   \end{enumerate}
\item Si la resposta és afirmativa, quina?
\item Creus que la Universitat de València hauria de mantenir el seu node de Mastodon?
   \begin{enumerate}
   \item Sí.
   \item No.
   \item NS/NC.
   \end{enumerate}

\item T'agradaria suggerir algun altre servei informàtic general que trobes a faltar?
\end{enumerate}

\end{document}

