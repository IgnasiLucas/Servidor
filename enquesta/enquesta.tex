\documentclass[a4paper,12pt]{article}
\usepackage[utf8]{inputenc}
\usepackage[T1]{fontenc}
\usepackage[catalan]{babel}
\usepackage{booktabs}
\usepackage[colorlinks=true]{hyperref}
\title{Enquesta sobre serveis informàtics generals per a la docència}
\hyphenation{Python}

\begin{document}
\maketitle
Un aspecte important de la transformació digital de la societat actual és l'ús creixent
dels llenguatges de programació, tant en àmbits professionals com educatius, a tots els
nivells i en qualsevol àrea. En la docència universitària, llenguatges com Python o R
estan passant de ser objectius d'aprenentatge en cursos especialitzats a ser un recurs
didàctic o vehicle d'ensenyança de qualsevol matèria i en qualsevol grau. Part de la
popularitat d'aquests llenguatges
de programació és conseqüència de poder-se expressar en text pla, un estàndard
universal fàcilment interpretable en qualsevol sistema operatiu. Aquesta
transparència dels materials didàctics elaborats en Python, R o gairebé qualsevol altre
llenguatge de programació facilita enormement que siguen compartits, reutilitzats i
actualitzats de manera col·lectiva i amb llicències obertes, tant per una comunitat docent
com per les i els estudiants. Altres tecnologies basades en el text pla (\LaTeX, git,
markdown, Jupyter notebooks, etc.) contribueixen a donar un caràcter obert a l'educació superior
i a desenvolupar les habilitats computacionals bàsiques de les i els estudiants.

Amb l'única finalitat de planificar la possible implementació de nous serveis informàtics que
faciliten l'adopció d'aquestes tecnologies, el Servei d'Informàtica necessita conéixer el
grau d'interés que susciten. Per aquest motiu, us demanem que contesteu les preguntes següents,
que no us ocuparan més de 5 minuts. Les respostes individuals seran completament anònimes i no seran
compartides amb tercers. Els estadístics de resum seran publicats a la pàgina web del Servei
d'Informàtica. 

\begin{enumerate}
\item Centre al qual pertany:
\item Categoria professional:
\item Gènere:
\item Edat:
   \begin{enumerate}
   \item Menys de 31.
   \item 31-40.
   \item 41-50.
   \item 51-60.
   \item Més de 60.
   \end{enumerate}
\end{enumerate}

Marca el grau de conformitat amb cadascuna de les afirmacions següents, entre 
1 (en complet desacord) i 5 (completament d'acord).

{\small
\begin{tabular}{|r|p{9cm}|c|c|c|c|c|}
\toprule
&&1&2&3&4&5\\
\midrule
5&Tinc bons coneixements d'almenys un llenguatge de programació.&&&&&\\
\midrule
6&L'ensenyança de les matèries en què participe podria millorar amb l'adopció d'un llenguatge de programació.&&&&&\\
\midrule
7&Participe o podria participar activament en la docència de pràctiques amb ordinador.&&&&&\\
\midrule
8&Estaria disposat o disposada a rebre formació en l'aplicació de la programació informàtica a la docència.&&&&&\\
\midrule
9&Les aules d'informàtica amb què comptem disposen de Linux i de tots els programes que necessitem.&&&&&\\
\midrule
10&La major part de programes informàtics que ensenyem a utilitzar són programari lliure.&&&&&\\
\midrule
11&Els i les estudiants poden instal·lar fàcilment als seus ordinadors els programes informàtics que utilitzem a l'aula.&&&&&\\
\midrule
12&Conec i utilitze amb certa freqüència documents dinàmics d'Rmarkdown en l'entorn d'RStudio \cite{RStudio}.&&&&&\\
\midrule
13&Conec i utilitze amb certa freqüència els quaderns Jupyter (Jupyter Notebooks) \cite{Barba2019}.&&&&&\\
\midrule
14&La UV hauria d'oferir un servidor d'RStudio perquè estudiants i professorat puguen connectar-se remotament.&&&&&\\
\midrule
15&La UV hauria d'oferir un servidor JupyterHub de quaderns Jupyter \cite{JupyterHub}.&&&&&\\
\midrule
16&La UV hauria d'oferir accés per \textsf{ssh} a un servidor de càlcul per a estudiants.&&&&&\\
\midrule
17&La UV hauria d'allotjar el seu propi servidor de repositoris de \textsf{git} (e.g., Gitlab) per a professorat i estudiants \cite{Paderborn2023,PereiraBraga2023}.&&&&&\\
\midrule
18&La UV hauria de mantenir un node propi d'una xarxa social federada, com Mastodon, i oferir comptes a tota la comunitat, igual que amb el correu electrònic \cite{Brembs2023,mastodon}.&&&&&\\
\bottomrule
\end{tabular}
}

Si vols suggerir algun altre servei, fer algun comentari o deixar la teua adreça electrònica per rebre més informació, pots fer-ho ací:
\vspace*{3cm}

\bibliography{../references}
\bibliographystyle{ieeetr}
\end{document}

