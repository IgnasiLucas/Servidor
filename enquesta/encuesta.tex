\documentclass[a4paper,12pt]{article}
\usepackage[utf8]{inputenc}
\usepackage[T1]{fontenc}
\usepackage[spanish]{babel}
\usepackage{booktabs}
\usepackage[colorlinks=true]{hyperref}
\title{Encuesta sobre servicios generales informáticos para la docencia}
\hyphenation{Python}

\begin{document}
\maketitle
Un aspecto importante de la transformación digital de la sociedad actual es el uso creciente
de los lenguajes de programación, tanto en ámbitos profesionales como educativos, a todos
los niveles y en cualquier área. En la docencia universitaria, lenguajes como Python o R
están pasando de ser objetivos del aprendizaje en cursos especializados a ser un recurso
didáctico o vehículo de enseñanza de cualquier materia, en cualquier grado. Parte de la
popularidad de estos lenguajes de programación es debida al hecho de ser expresados en
texto plano, un estándar universal fácilmente interpretable en cualquier sistema operativo.
Esta transparencia de los materiales didácticos elaborados en Python, R o en casi cualquier
otro lenguaje de programación facilita enormemente que sean compartidos, reutilizados y
actualizados de manera colectiva y con licencias abiertas, tanto por una comunidad docente
como por los y las estudiantes. Otras tecnologías basadas en el texto plano (\LaTeX, git,
markdown, Jupyter notebooks, etc.) contribuyen a dar un carácter abierto a la educación
superior y a desarrollar las habilidades computacionales básicas de los y las estudiantes.

Con la única finalidad de planificar una posible implementación de nuevos servicios
informáticos que faciliten la adopción de estas tecnologías, el Servicio de Informática
necesita conocer el grado de interés que suscitan. Por este motivo, os pedimos que
contestéis las siguientes preguntas, que no os ocuparán más de 5 minutos. Las respuestas
individuales serán completamente anónimas y no serán compartidas con terceros. Los estadísticos
de resumen serán publicados en la página web del Servicio de Informática.

\begin{enumerate}
\item Centro al que pertenece:
\item Categoría profesional:
\item Género:
\item Edat:
   \begin{enumerate}
   \item Menos de 31.
   \item 31-40.
   \item 41-50.
   \item 51-60.
   \item Más de 60.
   \end{enumerate}
\end{enumerate}

Marca el grado de conformidad con cada una de las afirmaciones siguientes, entre 1
(en completo desacuerdo) y 5 (completamente de acuerdo).

{\small
\begin{tabular}{|r|p{9cm}|c|c|c|c|c|}
\toprule
&&1&2&3&4&5\\
\midrule
5&Tengo buenos conocimientos de al menos un lenguaje de programación.&&&&&\\
\midrule
6&La enseñanza de las materias en las que participo podría mejorar con la adopción de un lenguaje de programación.&&&&&\\
\midrule
7&Participo o podría participar activamente en la docencia de prácticas con ordenador.&&&&&\\
\midrule
8&Estaría dispuesto o dispuesta a recibir formación en la aplicación de la programación informática a la docencia.&&&&&\\
\midrule
9&Las aulas de informática con las que contamos disponen de Linux y de todos los programas que necesitamos.&&&&&\\
\midrule
10&La mayor parte de programas informáticos que enseñamos a utilizar son software libre.&&&&&\\
\midrule
11&Los y las estudiantes pueden instalar fácilmente en sus ordenadores los programas informáticos que utilizamos en el aula.&&&&&\\
\midrule
12&Conozco y utilizo con cierta frecuencia documentos dinámicos de Rmarkdown en el entorno de RStudio \cite{RStudio}.&&&&&\\
\midrule
13&Conozco y utilizo con cierta frecuencia los cuadernos Jupyter (Jupyter Notebooks) \cite{Barba2019}.&&&&&\\
\midrule
14&La UV debería ofrecer un servidor de RStudio para que estudiantes y profesorado pueda conectarse remotamente.&&&&&\\
\midrule
15&La UV debería ofrecer un servidor JupyterHub de cuadernos Jupyter \cite{JupyterHub}.&&&&&\\
\midrule
16&La UV debería ofrecer acceso por \textsf{ssh} a un servidor de cálculo para estudiantes.&&&&&\\
\midrule
17&La UV debería alojar su propio servidor de repositorios de \textsf{git} (e.g., Gitlab) para estudiantes y profesorado \cite{Paderborn2023,PereiraBraga2023}.&&&&&\\
\midrule
18&La UV debería mantener un nodo propio de una red social federada, como Mastodon, y ofrecer cuentas a toda la comunidad, igual que con el correo electrónico \cite{Brembs2023,mastodon}.&&&&&\\
\bottomrule
\end{tabular}
}

Si quieres sugerir algún otro servicio, hacer algún comentario o dejar tu dirección electrónica para recibir más información, puedes hacerlo aquí:
\vspace*{3cm}

\bibliography{../references}
\bibliographystyle{ieeetr}
\end{document}
